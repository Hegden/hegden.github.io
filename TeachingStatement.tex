I believe that effective teaching is the key to inspire future researchers. 
As a teacher, I hope to teach with the goal of getting my students excited and thinking about the subject, much as my former colleagues in the industry and professors at IIT Madras and Purdue have gotten me inspired and thinking about topics in computer science. 

\paragraph{Teaching and mentoring experience}
I am currently serving as a TA to ECE295 (introduction to data science) and in the past, have served as a teaching assistant for ECE337 (Introduction to ASIC design), at Purdue. 
I was a lab instructor for ECE337 and handled 2 lab sections mentoring about 20 students in each section. 
During every lab session, I gave an oral presentation at the beginning of the session followed by one-on-one interaction with students. 
I also met students one-one-one during office-hour sessions and during mentoring of student teams on their project. 
I assisted students on their Verilog programming necessary for solving the assignments, graded their assignments and exams, and advised 8-10 student teams each term on their project work starting from conceptualizing to project presentation.
During the two terms that I served, I consistently received good feedback.  
I have also taken the lead in mentoring a final year undergraduate student on conducting research and this collaboration has resulted in a publication. 
At IIT Madras, I have served as a teaching assistant to courses CS3100 (Paradigms of programming), and CS1300 (introduction to computer science).
Such ppportunities to interact, advise, and also learn are a major force in driving me towards an academic career.

\paragraph{Teaching style and philosophy}
My approach towards teaching is to build knowledge as opposed to depositing knowledge, and my philosophy is that one should actively seek knowledge with an open-minded approach and that learning should never stop.  
I strongly believe that students must imbibe skills related to working in teams as early as possible in order to be prepared for the future.
Working in teams is the most essential skill one must possess either in industry or in academic settings while collaborating with colleagues and yet, is often ignored in classical teaching styles.
My most memorable moments were those involving joint deveopment (especially testing and debugging) of projects in the industry and academia (course projects and my masters thesis project).
When implemented effectively, team work not only engages students with one another but also with the teacher. 
And I believe engagement is crucial to get students excited about the subject.
In this regard, I would structure my syllabus around getting students engaged: asking questions during lectures and encouraging groups of students to discuss answers loud, framing assignments and projects involving teams.
I would also encourage students actively seek me by having an open-door policy and be a friendly ear to their concerns coursework or otherwise. 
This way, I would promote interaction and also get feedback.
I will also reflect on my practice by soliciting mid-term feedback.

I expect my teaching style to remain the same for both undergraduate and graduate students. 
At the graduate level, I believe that emphasis on seminars and projects are ideal for fostering new ideas and realizing those ideas into implementations.
Besides, seminars and projects also serve to enhance presentation and communication skills of students.


\paragraph{Course qualifications}
At the undergraduate level, I can teach courses on parallel computing (including introducing languages such as OpenMP, MPI, OpenCL), data science, compilers, data structures, and ASIC design.
I can also teach graduate level courses on parallel programming and compilers.  







%I know when I am effectively teaching when students show interest: i) they ask questions that probe, ii) they read related topics beyond what is presented in class, and iii) they actively seek a hands-on approach. 
%What beliefs / theories mark my teaching?
%-Once basic concepts are introduced, be a channel of ideas exchange rather than ?.
%How do I maintain positive relationship with students?
%Be available for students concerns sometime beyond course topics. Provide students with flexibility in adapting to course content, have open door policy. 
%How do I collaborate with others?
%First, respect every other person as a human being, Show genuine interest in their research. Ask relevant questions.
%How do I create a supportive learning environment?
%Be friendly or available to students? concerns. Be adaptable in framing course content: pace the teaching such that all the students are up to mark. 
%What approach or strategy has worked especially well in my discipline?
%Hands-on practical training, programming. Exposure to real-world applications.
%What are my most important learning goals?
%Advancing my students knowledge is primary.
%How have/can I make a difference in the life of my students?
%Try to convey to my students that one should keep an open mind in life and actively seek knowledge. Learning should never stop. 
  

