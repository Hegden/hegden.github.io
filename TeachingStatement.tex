I believe that effective teaching is the key to inspire future researchers. 
As a teacher, I hope to teach with the goal of getting my students excited and thinking about the subject, much as my former colleagues in the industry and professors at Purdue and IIT Madras have gotten me inspired and thinking about topics in computer science and engineering. 

\paragraph{Teaching and mentoring experience}
At Purdue, I am currently serving as a teaching assistant (TA) to Introduction to Data Science. 
Unlike typical offerings of Data Science-related courses, I think the course at Purdue provides a holistic approach because
it includes three components: i) data processing (programming algorithms, analysis and visualization), ii) data management (from a library sciences perspective), and iii) ethical issues of handling data (from a philosophical sciences perspective).
As a TA in this course, I assist students with learning skills necessary to succeed in the first component only.
In this course we also use an online discussion forum called Piazza, which makes instructors available to students virtually 24 hours a day.
I also served as a TA to Introduction to ASIC design at Purdue. 
Based on the experience in this course, where we did not use Piazza, I think tools like Piazza are necessary to see an increased student involvement, especially when the class size is large and the TA office hours are limited, and/or a hands-on (lab/programming) component is involved. 
In both these courses, I have primarily been assisting students in labs on their programming skills (Verilog and Python), and project work.
In the ASIC design course, I lectured at the beginning of every lab session, and mentored several student teams each term on their project work involving conceptualization, design, execution, and presentation.
During the two terms that I taught, I consistently received good feedback.  
%I have closely interacted with students during lab sessions, office-hours, and during mentoring of student teams on their project. 
Recently, I also took the lead in mentoring an undergraduate student on a research project. 
We met bi-weekly to discuss literature and project implementation strategy.   
A technical report of this project is available and an updated version is under submission to a conference. 
At IIT Madras, I served as a TA to Paradigms of Programming, and Introduction to Computer Science and Engineering.
I did not get feedback for these courses, since a mechanism to evaluate TAs was not available when I taught these courses.
These experiences to interact, advise, and also learn have been a major force in driving me towards an academic career.

\paragraph{Teaching style and philosophy}
Having both studied and taught at Purdue and IIT Madras, I believe that I bring a unique perspective to teaching. 
In creating an effective learning atmosphere, I would like to bring the pedagogical innovations experienced at Purdue (novel tools, different perspectives to curriculum etc.) into my teaching.
I have personally found learning (of a course in academia, or of an application domain in the industry such as {\em location-based services}) exciting when it engaged me. E.g.
presenting a technical paper of my interest that is related to a course, debugging a wireless protocol implementation by physically moving around the hallway, a well-prepared instructor asking questions and encouraging groups of students to discuss answers out loud, or even framing quizzes with an opening general knowledge question regarding topic of the season (politics, movies).  
I would include all these attributes into my teaching style to engage students and get them excited: I would use tools such as Piazza, be well prepared, ask questions during lectures and encourage groups of students to discuss answers out loud, frame assignments and projects involving teams, and depending upon an undergraduate or graduate level course, encourage teams of students to present technical papers of their choice. 
The latter activity is especially important in graduate level teaching for fostering new ideas and realizing those ideas into implementations.
I would also encourage students to actively seek me by having an open-door policy and be a friendly ear to their concerns about coursework or otherwise. 
I would reflect on my practice by soliciting feedback (named/anonymous) through physical or virtual channels, for instance, when an instructor at Purdue created `Piazza channels'' to solicit feedback for each lecture during an introductory offering of an advanced graduate level course.
I believe that working in teams and effective communication are the most essential skills to succeed in industry or in academic settings and yet are often ignored in traditional teaching styles.
Through discussion groups, projects and presentations, I hope to encourage students to develop these skills. 
In doing all this, I would expect my teaching style to reflect building knowledge as opposed to depositing knowledge, and a philosophy that one should actively seek knowledge with an open-minded approach and that learning should never stop.  

\paragraph{Course qualifications}
%At the undergraduate level, I can teach courses on parallel computing (including introducing languages such as OpenMP, MPI, OpenCL), data science, compilers, data structures, and ASIC design.
%#I can also teach graduate level courses on parallel programming and compilers.  
At the undergraduate level, I can teach courses on basic data-structures and programming, compilers, and data science (e.g. CS101, CS113, CS193, CS2xD from the list of core courses). At the graduate level, I can teach courses on parallel computing and compilers.
I believe that it is necessary to introduce concepts on parallelism early in the curriculum considering the computing power demanded by modern applications, and the pervasiveness and availability of parallel hardware. 
In this regard, I am interested in developing and updating courses on parallel programming using examples from topical domains such as Machine Learning and BigData.







%I know when I am effectively teaching when students show interest: i) they ask questions that probe, ii) they read related topics beyond what is presented in class, and iii) they actively seek a hands-on approach. 
%What beliefs / theories mark my teaching?
%-Once basic concepts are introduced, be a channel of ideas exchange rather than ?.
%How do I maintain positive relationship with students?
%Be available for students concerns sometime beyond course topics. Provide students with flexibility in adapting to course content, have open door policy. 
%How do I collaborate with others?
%First, respect every other person as a human being, Show genuine interest in their research. Ask relevant questions.
%How do I create a supportive learning environment?
%Be friendly or available to students? concerns. Be adaptable in framing course content: pace the teaching such that all the students are up to mark. 
%What approach or strategy has worked especially well in my discipline?
%Hands-on practical training, programming. Exposure to real-world applications.
%What are my most important learning goals?
%Advancing my students knowledge is primary.
%How have/can I make a difference in the life of my students?
%Try to convey to my students that one should keep an open mind in life and actively seek knowledge. Learning should never stop. 
  

